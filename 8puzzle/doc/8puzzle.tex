\documentclass[11pt]{article}
\usepackage[utf8]{inputenc}
\usepackage[czech]{babel}
\usepackage{amsmath}

\author{Tomáš Maršálek}
\title{Semestrální práce z předmětu KIV/UIR}
\date{\today}

\begin{document}
\begin{titlepage}
\maketitle
\end{titlepage}

\section{Zadání}
V programovacím jazyce C zpracujte jednoduchý výukový program, který plně
demonstruje nalezení řešení následující úlohy: Jde o převedení hlavolamu \uv{8}
(při respektování pořadí aplikace produkčních pravidel uvedeného na přednášce)
% TODO nějaký pikčůry
metodou prohledávání grafu s využitím heuristické funkce
\begin{align*}
\hat{f}(n_i)~&=~\hat{g}(n_i) + \hat{h}(n_i) \\
\hat{g}(n_i)~&=~d(n_i)~~~~(d\acute{e}lka~cesty~z~n_0~do~n_i) \\
\hat{h}(n_i)~&=~P(n_i) + 3 Q(n_i) \\
\end{align*}

\begin{itemize}
	\item $P(n_i)$ je součet vzdáleností každého kamene hlavolamu od svého
          cílového místa (v možných posuvech).
	\item $Q(n_i)$ je míra porušení pořadí kamenů zahrnutá tak, že
	\begin{itemize}
		\item přičítáme hodnotu 2 za každý kámen nenacházející se ve středu
              pole a jenž není následován správným kamenem,
		\item za kámen ve středu pole přičítáme 1.
	\end{itemize}
\end{itemize}

\section{Analýza problému}
\section{Implementace}
\section{Uživatelská příručka}
\section{Řešení úlohy}
\section{Závěr}

\end{document}
