\documentclass[12pt]{article}
\usepackage[utf8]{inputenc}
\usepackage[czech]{babel}
\usepackage{graphicx}
\usepackage{amsmath}
\usepackage{amssymb}


\begin{document}
\begin{titlepage}
\begin{center}
	\mbox{} \\[3cm]
	\huge{Semestrální práce z předmětu KIV/UIR} \\[2.5cm]
	\Large{Tomáš Maršálek} \\
	\large{marsalet@students.zcu.cz} \\[1cm]
	\normalsize{\today}
\end{center}
\thispagestyle{empty}
\end{titlepage}

\section{Zadání}
V programovacím jazyce C zpracujte jednoduchý výukový program, který plně
demonstruje nalezení řešení následující úlohy: Jde o převedení hlavolamu \uv{8}
(při respektování pořadí aplikace produkčních pravidel uvedeného na přednášce)

\vspace{.3cm}

\begin{center}
\begin{minipage}[b]{.3\textwidth}
\centering
ze stavu \\
\vspace{.1cm}
\begin{LARGE}
\begin{tabular}{|ccc|}
\hline
1 &  $\square$ & 3 \\
6 & 2 & 4 \\
8 & 7 & 5 \\
\hline
\end{tabular}
\end{LARGE}
\end{minipage}
\begin{minipage}[b]{.2\textwidth}
{\large $R_k$} \\
$\longrightarrow$
\end{minipage}
\begin{minipage}[b]{.3\textwidth}
\centering
do stavu \\
\vspace{.1cm}
\begin{LARGE}
\begin{tabular}{|ccc|}
\hline
1 & 2 & 3 \\
8 & $\square$ & 4 \\
7 & 6 & 5 \\
\hline
\end{tabular}
\end{LARGE}
\end{minipage}
\end{center}

\vspace{.3cm}

metodou prohledávání grafu s využitím heuristické funkce
\begin{align*}
\hat{f}(n_i)~&=~\hat{g}(n_i) + \hat{h}(n_i) \\
\hat{g}(n_i)~&=~d(n_i)~~~~(d\acute{e}lka~cesty~z~n_0~do~n_i) \\
\hat{h}(n_i)~&=~P(n_i) + 3 Q(n_i) \\
\end{align*}

\begin{itemize}
	\item $P(n_i)$ je součet vzdáleností každého kamene hlavolamu od svého
          cílového místa (v možných posuvech).
	\item $Q(n_i)$ je míra porušení pořadí kamenů zahrnutá tak, že
	\begin{itemize}
		\item přičítáme hodnotu 2 za každý kámen nenacházející se ve středu
              pole a jenž není následován správným kamenem,
		\item za kámen ve středu pole přičítáme 1.
	\end{itemize}
\end{itemize}


\section{Analýza úlohy}
Cílem úlohy je zjistit posloupnost tahů, která řeší výše uvedený hlavolam.  Při
hledání jednotlivých tahů je možné postupně vypisovat postup algoritmu.

Řešení tohoto hlavolamu je možné provést prohledáváním stavového prostoru, tzn.
všech možných pozic kamenů. Jednou takovou možností je použití slepých
strategií pro prohledávání grafu stavového prostoru (prohledávání do hloubky,
prohledávání do šířky), nicméně efektivnější strategií, co se týče velikosti
podmnožiny prohledaných stavů hlavolamu, jsou cílené strategie. Pro tuto úlohu
je zvolena strategie heuristického prohledávání grafu stavového prostoru s výše
uvedenou heuristickou funkcí.

Oproti slepým strategiím je výhoda v menším počtu prohledaných stavů, na druhou
stranu může být nevýhodou složitý výpočet heuristické funkce.

\section{Implementace}
Protože je zvolená cílená strategie prohledávání, primární datovou strukturou
není graf ani strom, ale prioritní fronta. V každém kroku algoritmu nás zajímá
stav hlavolamu s nejnižším ohodnocením. Grafová struktura stavového prostoru je
tvořena až dodatečně pomocí rodič-potomek ukazatelů. Aby nedocházelo k cyklení
v případě neexistujícího řešení a omezení zbytečného generování stejných stavů,
je implementována datová struktura množina, ve které jsou uchovávány
vygenerované stavy. Nikdy se tedy nestane, že by se stejný stav vygeneroval
vícekrát.

Jazyk implementace je, dle zadání, ANSI C. Je zajištěna správná manipulace s
pamětí, nedochází k paměťovým únikům a při skončení programu je rovněž čistá
halda.

\subsection{Prioritní fronta}
I když je velikost zadané úlohy poměrně malá, je prioritní fronta
implementována jako binární halda, přitom by stačilo použít jednoduchý seznam.

\subsection{Množina}
Implementace pomocí rozptylové tabulky s řetězením záznamů při kolizi.


\section{Uživatelská příručka}
Program je odevzdán jako zip soubor obsahující zdrojové soubory a makefile pro
přeložení pomocí gcc.

\subsection{Uživatelské módy}
Bez specifikování jakýchkoliv dobrovolných argumentů je uživateli pouze
představena posloupnost řešících tahů. Je zde ale možnost podívat se jak je
generován stavový prostor s možností krokování jednotlivých iterací nebo ne.

\paragraph{}
Argument (-h) vypíše do konzole nápovědu a skončí program.

\paragraph{}
Volitelný verbózní mód (-v) vypisuje číslo iterace, hloubku právě vytaženého
stavu a všechny dosud neobjevené potomky, které z tohoto stavu vedou dál. Pro
každý stav je zobrazena hodnota heuristické funkce.

\paragraph{}
Dalším volitelným módem je krokování (-s), které umožňuje uživateli programu
krokovat jednotlivé iterace prohledávání v případě, že je zvolen verbózní mód.

\paragraph{}
Dodatečně je implementována možnost uložení výsledného výpisu do souboru,
standardně pomocí přepínače (-o) a názvu souboru.


\section{Závěr}
Cílem bylo splnit zadání co možná nejpřesněji. 

Překvapením byl nízký počet kroků nutných k nalezení řešení a výsledná krátká
posloupnost tahů vedoucích k řešení. Program je tedy možná řešen zbytečně
složitě, nicméně má potenciál být rozšířen na stejnou úlohu o větších
rozměrech.

Jednoduchým rozšířením by byla implementace možnosti zadávat vlastní počáteční
a konečný stav hlavolamu, nicméně zadání tohle nevyžadovalo.

Možné vylepšení by bylo zavedení grafického výstupu namísto tisknutí do
konzole, například formou animace generování stavového grafu.

\end{document}
